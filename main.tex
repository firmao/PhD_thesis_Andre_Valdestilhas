\documentclass[ oneside,openright,titlepage,numbers=noenddot,headinclude,%1headlines,% letterpaper a4paper
                footinclude=true,cleardoublepage=empty,abstractoff, % <--- obsolete, remove (todo)
                BCOR=5mm,paper=a4,fontsize=11pt,%11pt,a4paper,%
                 ngerman,american,%
                ]{scrreprt}
                
%\documentclass[a4paper,twoside,abstracton,12pt,BCOR=15mm]{scrreprt}

%********************************************************************
% Note: Make all your adjustments in here
%*******************************************************
\input{classicthesis-config}

\usepackage[automark,headsepline]{scrpage2}
%\usepackage{blindtext}
%\usepackage[colorlinks=true,linkcolor=black,citecolor=black,urlcolor=black]{hyperref}
%\usepackage[utf8]{inputenc}
%\usepackage[english]{babel}
%\usepackage{amsmath,amssymb}
%\usepackage{amsthm}
%\usepackage{verbatim}
%\usepackage{color,graphicx}
%\usepackage{microtype}
\usepackage{import}

\usepackage{qname}
\setprefix{owl}{http://www.w3.org/2002/07/owl\#}
\setprefix{dbpedia}{http://dbpedia.org/resource/}

%\newcommand{\xmark}{\ding{55}}%

\newcommand{\E}{{\mathcal{E}}}
\newcommand{\F}{{\mathcal{F}}}
\newcommand{\ds}{{\ensuremath{D_s}}}
\newcommand{\dt}{{\ensuremath{D_t}}}

\newcommand{\func}[2]{\textnormal{#1}(#2)}
\newcommand{\f}[1]{\func{f}{#1}}
\newcommand{\h}[1]{\func{h}{#1}}

\DeclareRobustCommand{\hr3}{{$\mathcal{HR}^3$}}

\usepackage{url}
\usepackage{todonotes}
\usepackage{booktabs}
\usepackage{pgfplots}
\usepackage{aurl}
\usepackage{upgreek}
\usepackage{subfigure}
\usepackage{amsfonts}
\usepackage{amssymb}
\usepackage{amsmath}
\usepackage{graphicx}
\usepackage{tabularx}
\usepackage{algorithm}
\usepackage{algpseudocode}
\usepackage{hyperref}
\usepackage{cleveref} 
\usepackage{multicol}
\usepackage[utf8]{inputenc}
\usepackage[T1]{fontenc}
\usepackage[scaled=0.85]{beramono}
\usepackage{listings}
\usepackage{filecontents}
\lstset{language=SQL,morekeywords={PREFIX,java,rdf,rdfs,url}}
\usepackage{pifont}% http://ctan.org/pkg/pifont

\usepgfplotslibrary{groupplots}
\usetikzlibrary{patterns}

\usetikzlibrary{graphs}

\usepackage{makeidx}
\makeindex

\pagestyle{scrheadings}
\automark[section]{chapter}

% Pflichtbestandteile der Arbeit nach Promotionsordnung:
% - vorgeschriebenes Deckblatt
% - CV
% - bibliographische Daten
% - Selbständigkeitserklärung
% - Inhaltsverzeichnis
% - Literaturverzeichnis

% avoid placing figures/screenshots on a page by itself with no text
\renewcommand{\topfraction}{0.85}
\renewcommand{\textfraction}{0.1}

\begin{document}

\renewcommand{\thepage}{\Roman{page}}

% Deckblatt entspricht Promotionsordnung
\begin{titlepage}
%\pagestyle{empty}

\begin{center}
\large

$\quad$\\
$\quad$\\
%{\Huge Exploring the quality of linksets on linked data}\\
%{\Huge Identifying, Relating and Querying datasets on the Web of Data}\\ 
%{\Huge Identifying, Integrating and Querying datasets on the Web of Data}\\
{\Huge Identifying, Relating and Querying Large Heterogeneous RDF Sources}\\
%(draft revision \svnrev{})\\
\vspace{1cm}
Der Fakult\"at f\"ur Mathematik und Informatik\\
der Universit\"at Leipzig\\
eingereichte\\
\vspace{1cm}
{\Huge DISSERTATION}\\
\vspace{1cm}
zur Erlangung des akademischen Grades\\
\vspace{1cm}
{\Large \scshape{Doctor Rerum Naturalium}}\\
(Dr. rer. nat.)\\
\vspace{1cm}
im Fachgebiet\\
{\large Informatik}\\
\vspace{1cm}
vorgelegt\\
\vspace{1cm}
{von \textbf{Dipl.-Inf. Andre Valdestilhas}}\\
\vspace{1cm}
geboren am 14. Feb. 1980 in Sao Bernardo do Campo-SP, Brazil\\
\vspace{1cm}
Leipzig, den \the\day.\the\month.\the\year

\vfill
%\includegraphics[width=0.30\textwidth]{siegel_grau.eps}
%\includegraphics[width=0.30\textwidth]{424_m_gr.EPS}

\end{center}
\cleardoublepage
\end{titlepage}

\tableofcontents
\newpage
\begingroup 
    \let\clearpage\relax
    \let\cleardoublepage\relax
    \let\cleardoublepage\relax
    %*******************************************************
    % List of Todos
    %*******************************************************    
    %\phantomsection 
    %\refstepcounter{dummy}
    %\addcontentsline{toc}{chapter}{\listfigurename}
    %\pdfbookmark[1]{\listoftodos}{lof}
    %\listoftodos

    %\vspace*{8ex}
    
    %*******************************************************
    % List of Figures
    %*******************************************************    
    \addcontentsline{toc}{chapter}{List of Figures}
    \pdfbookmark[1]{\listfigurename}{lof}
    \listoffigures
    \vspace*{8ex}

    %*******************************************************
    % List of Tables
    %*******************************************************
    \newpage
    \addcontentsline{toc}{chapter}{List of Tables}
    \pdfbookmark[1]{\listtablename}{lot}
    \listoftables
    \vspace*{8ex}
    
    %----------------------------------------------------------------------------------------
	%	List of Algorithms
	%----------------------------------------------------------------------------------------
% \newpage
% 	\refstepcounter{dummy}
% 	\let\oldnumberline\numberline
% 	\renewcommand{\numberline}{Algorithm~\oldnumberline}
% 	\pdfbookmark[1]{List of Algorithms}{loa} % Bookmark name visible in a PDF viewer
% 	\listofalgorithms
% 	\vspace*{8ex}
% \endgroup


% main content goes here
\newpage
\chapter{Abstract}
\import{sections/}{Abstract.tex}

\newpage
\chapter{Publications}
\import{sections/}{publications.tex}

\newpage
\chapter{ACKNOWLEDGMENTS}
\import{sections/}{ACKNOWLEDGMENTS.tex}

\newpage
\chapter{Introduction}

\renewcommand{\thepage}{\arabic{page}}
\setcounter{page}{1}

\import{sections/}{Introduction.tex}

\newpage
\chapter{Basic Concepts and Notation} \label{ch:preliminaries}
\import{sections/}{preliminaries.tex}

\newpage
\chapter{State of the Art} \label{ch:soa}
\import{sections/}{soa.tex}

\newpage
\chapter{Identifying Datasets in Large Heterogeneous RDF sources} \label{ch:wimu}
\import{sections/}{wimu.tex}

\newpage
\chapter{Relating Large amount of RDF datasets} \label{ch:lodindex}
\import{sections/}{LodIndex.tex}

\newpage
\chapter{Obtaining Similar Resources using String Similarity} \label{ch:mfkc}
\import{sections/}{MFKC.tex}

\newpage
\chapter{Heterogeneity in DBpedia Identifiers} \label{ch:dbpediasameas}
\import{sections/}{DBpediaSameAs.tex}

\newpage
\chapter{Detection of Erroneous Links in Large-Scale RDF datasets} \label{ch:cedal}
\import{sections/}{CEDAL.tex}

\newpage
\chapter{Querying Large Heterogeneous RDF Datasets} \label{ch:wimuq}
\import{sections/}{wimuq.tex}

\newpage
\chapter{Conclusion} \label{ch:conclusion}
\import{sections/}{Conclusion.tex}

\appendix

\newpage
\chapter{Appendix}
\import{sections/}{Applications.tex}

\newpage
\chapter{CV} \label{ch:cv}
\import{sections/}{CV.tex}

%\chapter{Expose} \label{ch:expose}
%\import{sections/}{Expose.tex}
\newpage
\bibliographystyle{apalike}
\bibliography{aksw,personal}

% \clearpage\phantomsection\addcontentsline{toc}{chapter}{I do not know}\chapter*{I simply do not know}

% I do not know I do not know I do not know I do not know I do not know I do not know I do not know I do not know I do not know I do not know I do not know I do not know I do not know I do not know I do not know I do not know I do not know I do not know I do not know I do not know I do not know I do not know I do not know I do not know I do not know.

\vspace{20pt}\noindent
Leipzig, den \the\day.\the\month.\the\year

\vspace{80pt}\noindent
Andre Valdestilhas

\end{document}
